\documentclass{article}
% Formatting-this makes it a full page
\usepackage[margin=1in]{geometry}
%This allows you to use most math symbols
\usepackage{amsmath,amsfonts,amssymb,mathtools}

%This package, and the following lines, allow you to define lemmas, corollaries, theorems, and examples.  
\usepackage{amsthm}
\newtheorem{theorem}{Theorem}
\newtheorem{corollary}{Corollary}[theorem]
\newtheorem{lemma}{Lemma}[theorem]
\newtheorem{proposition}{Proposition}[theorem]
\newtheorem{example}{Example}[theorem]

\title{Math 163: Homework 1}
\author{Fangzheng Yu}
\date{\today}
\begin{document}
%Makes the title show up!

	\maketitle
        \begin{theorem}\label{thm:theTheoremIndeed}
                The sum of two odd integers is an even integer.
        \end{theorem}
        \begin{proof}
                Let $x$ and $y$ be odd integers\\
                Then there is an integer \(a\), such that
                $x = 2a + 1$\\
                Then there is an integer \(b\), such that
                $y = 2b + 1$\\
                Then,
                \[x + y = (2a + 1) + (2b + 1)\]
                By commutativity:
                \[x + y = 2a + 2b + 2\]
                By distributive law:
                \[x + y = 2(a + b + 1)\]
                Because
                $a, b, 1 \in \mathbb{Z}$\\
                Therefore
                $a + b + 1 \in \mathbb{Z}$ (by closure under addition of the integers)\\
                Thus, $x + y$ is an even integer by definition of even.
        \end{proof}
\end{document}